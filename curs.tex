\documentclass[a4paper,12pt]{article}
\usepackage[utf8x]{inputenc}
\usepackage[romanian]{babel}
%\usepackage[T2A]{fontenc}
\usepackage{amsmath}
\usepackage{amsfonts}
\usepackage{amssymb}
\usepackage[hidelinks]{hyperref}
%===============================================================================
\usepackage{ntheorem}
  \theoremstyle{change}
  \theorembodyfont{\upshape}
  \newtheorem{theorem}{Teoremă}[section]
  \newtheorem{definition}[theorem]{Definiție}
  \newtheorem{example}[theorem]{Exemplu}
  \newtheorem{proposition}[theorem]{Propoziție}
  \newtheorem{corollary}[theorem]{Corolar}
%  \newenvironment{proof}{{\bf Demonstrație:} }{}
  %\newtheorem{exercise}[theorem]{Exercițiu}
\newenvironment{proof}[1][Proof]{\begin{trivlist}
\item[\hskip \labelsep {\bfseries #1}]}{\end{trivlist}}  
  \newtheorem{question}[theorem]{\^{I}ntrebare}
  \newtheorem{problem}[theorem]{Problemă}
  \newtheorem{algorithm}[theorem]{Algoritm}
%===============================================================================
\usepackage{verbatim}
\usepackage{graphicx}
\usepackage{hyperref}
\usepackage{color}
\usepackage{caption}
\usepackage{subcaption}
%===============================================================================
\usepackage{titlesec}
  \titleformat{\chapter}
    {\normalfont\huge\bfseries}{}{20pt}{\Huge}
  \titlespacing*{\chapter}{0pt}{50pt}{40pt}
  \titleclass{\section}{straight}
\usepackage{xargs}
\usepackage{minibox}% http://tex.stackexchange.com/questions/8680/how-can-i-insert-a-newline-in-a-framebox
  \renewcommandx\minibox[3][1=l, 2=c]{%
    \begin{tabular}[#2]{@{}#1@{}}
      #3
    \end{tabular}%
  }
%===============================================================================

%===============================================================================
\usepackage{exercise}
  \renewcommand{\ExerciseName}{}
  \renewcommand{\ExerciseListName}{}
  \renewcommand{\ExerciseHeaderTitle}{\ExerciseTitle}
  \renewcommand{\ExerciseHeader}{\textbf{
                \ExerciseName\ExerciseHeaderNB.\ExerciseHeaderTitle
                \ExerciseHeaderOrigin}}
  \renewcommand{\ExerciseHeader}{\textbf{
                \ExerciseName\ExerciseHeaderNB.\ExerciseHeaderTitle}}
  \renewcommand{\ExerciseListHeader}{\ExerciseHeaderDifficulty%
              \textbf{\ExerciseListName\ \ExerciseHeaderNB.%
              \ \ExerciseHeaderTitle}%
              \ExerciseHeaderOrigin\ignorespaces}
  \renewcommand{\ExerciseListHeader}{\ExerciseHeaderDifficulty%
              \textbf{\ExerciseListName\ \ExerciseHeaderNB.%
              \ \ExerciseHeaderTitle}%
              \ignorespaces}
  \setlength{\ExerciseSkipBefore}{0\baselineskip}
  \setlength{\ExerciseSkipAfter}{0\baselineskip}
  \setlength{\Exesep}{0\baselineskip}
  \setlength{\Exetopsep}{0\baselineskip}
  \setlength{\Exeleftmargin}{25em}
  \setlength{\QuestionBefore}{0\baselineskip}
%===============================================================================

\newcommand{\N}[1]{\mathbb{#1}}




\title{Elemente de analiză funcțională\\Note de curs}
\author{Radu N. Dumbraveanu}

\begin{document}

\maketitle

\tableofcontents

%===============================================================================
\section{Săptămâna 7}
%===============================================================================

%-------------------------------------------------------------------------------
\subsection{Obiectul analizei funcționale}
%-------------------------------------------------------------------------------

% Spaţii metrice. Exemple. Inegalităţile Young, Holder, Minkowski.

%-------------------------------------------------------------------------------
\subsection{Spații metrice}
%-------------------------------------------------------------------------------

%-------------------------------------------------------------------------------
\subsection{Inegalităţile Young, H\"older, Minkowski}
%-------------------------------------------------------------------------------

% Mulţimi închise şi mulţimi deschise. Spaţii metrice separabile.

%-------------------------------------------------------------------------------
\subsection{Mulțimi deschise}
%-------------------------------------------------------------------------------

%-------------------------------------------------------------------------------
\subsection{Mulțimi închise}
%-------------------------------------------------------------------------------

%-------------------------------------------------------------------------------
\subsection{Spații metrice separabile}
%-------------------------------------------------------------------------------

%===============================================================================
\section{Săptămâna 8}
%===============================================================================

% Convergenţa într-un spaţiu metric. Şiruri fundamentale.

%-------------------------------------------------------------------------------
\subsection{Convergența în metrică}
%-------------------------------------------------------------------------------

%-------------------------------------------------------------------------------
\subsection{Șiruri fundamentale}
%-------------------------------------------------------------------------------

% Spaţii metrice complete. Exemple.

%-------------------------------------------------------------------------------
\subsection{Spații complete}
%-------------------------------------------------------------------------------

\begin{theorem}\label{thm:subspatiu-complet}
Fie $X$ un spațiu metric complet și $Y$ un subspațiu al lui $X$. Atunci $Y$ este complet dacă și numai dacă $Y$ este închis în $X$.
\end{theorem}

%-------------------------------------------------------------------------------
\subsection{Spații complete remarcabile}
%-------------------------------------------------------------------------------

\paragraph{Spațiile $\mathbb{R}^n$ și $\mathbb{C}^n$} Spațiul euclidian $\mathbb{R}^n$ și spațiul unitar $\mathbb{C}^n$ sunt complete.

\begin{proof}[Într-adevăr]
Vom considera mai întâi spațiul $\mathbb{R}^n$.
Și reamintim că metrica pe $\mathbb{R}^n$ (metrica euclidiană) este definită prin 

\[ d(x,y) = \left(\sum_{i=1}^n (\xi_i - \eta_i)^2\right)^\frac{1}{2} \]

\noindent unde $x=(\xi_i)$ și $y=(\eta_i)$. 

Fie $(x_m)$ un șir Cauchy în $\mathbb{R}^n$, cu $x_m=(\xi_1^{(m)},\xi_2^{(m)},...,\xi_n^{(m)})$. Atunci pentru orice $\varepsilon$ pozitiv, există un rang $m_{\varepsilon}$ astfel încît pentru toți termenii $x_m$ și $x_p$ de rang mai mare decît $m_{\varepsilon}$ ($m,p>m_{\varepsilon}$) să avem

\begin{equation}\label{l1}
d(x_m,x_p) = \left(\sum_{i=1}^n (\xi_i^{(m)}-\eta_i^{(p)})^2\right)^\frac{1}{2} < \varepsilon.
\end{equation}

Ridicînd la pătrat ambele părți ale inegalității obținem

\[ \sum_{i=1}^n (\xi_i^{(m)}-\eta_i^{(p)})^2 < \varepsilon^2. \]

Întrucît termenii sumei sînt nenegativi reiese că pentru orice $m,p>m_\varepsilon$ și $i=\overline{1,n}$ avem

\[ (\xi_i^{(m)}-\eta_i^{(p)})^2 < \varepsilon^2 \text{ și respectiv } |\xi_i^{(m)}-\eta_i^{(p)}| < \varepsilon. \]

Din ultima relație rezultă că pentru fiecare $i=\overline{1,n}$,  șirul de numere reale $\xi_i^{(1)}, \xi_i^{(2)}, ..., \xi_i^{(m)}, ...$ este un șir Cauchy. Acest șir este convergent, conform criteriului Cauchy de convergență a șirurilor numerice: $\xi_i^{(m)}\to \xi_i$ cînd $m\to\infty$. Folosind aceste $n$ limite definim $x=(\xi_1,\xi_2,...,\xi_n)$. Evident, $x\in\mathbb{R}^n$. Trecînd la limită în \eqref{l1} cu $p\to\infty$ pentru $m>m_\varepsilon$ avem

\[ d(x_m,x)\leq\varepsilon. \]

Așadar $x$ este limita șirului $(x_m)$ și întrucît acest șir a fost ales arbitrar spațiul $\mathbb{R}^n$ este complet.

Completitudinea spațiului $\mathbb{C}^n$ se demonstrează analog.

\end{proof}

\paragraph{Spațiul $l^\infty$} Spațiul $l^\infty$ este complet.

\begin{proof}[Într-adevăr]

Fie $(x_m)$ un șir Cauchy în $l^\infty$, unde $x_m=(\xi_1^{(m)},\xi_2^{(m)},...)$. Întrucît metrica pe $l^\infty$ este definită prin

\[ d(x,y) = \sup_{i=1}^{\infty} |\xi_i - \eta_i| \]

\noindent unde $x=(\xi_i)$ și $y=(\eta_i)$, iar $(x_m)$ este Cauchy, pentru orice $\varepsilon$ pozitiv, există un rang $n_{\varepsilon}$ astfel încît pentru toți termenii $x_m$ și $x_n$ de rang mai mare decît $m_{\varepsilon}$ ($m,n>m_{\varepsilon}$) să avem

\[ d(x_m,x_n) = \sup_{i=1}^{\infty} |\xi_i^{(m)} - \xi_i^{(n)}| < \varepsilon. \]

Deoarece supremul este luat de la elemente nenegative reiese că pentru orice $m,n>m_\varepsilon$ și $i$ fixat avem

\[
\label{l2}
|\xi_i^{(m)} - \xi_i^{(n)}| < \varepsilon.
\]

Din ultima relație reies că pentru fiecare $i$ șirul de numere reale (complexe) $\xi_i^{(1)}, \xi_i^{(2)}, ..., \xi_i^{(m)}, ...$ este un șir Cauchy. Acest șir converge conform criteriului Cauchy de convergență a șirurilor numerice: $\xi_i^{(m)}\to \xi_i$ cînd $m\to\infty$. Folosind aceste limite infinite definim $x=(\xi_1,\xi_2,...)$. Vom arăta că $x\in l^\infty$ și $x_m\to x$.

Trecînd la limită în \eqref{l2} cu $n\to\infty$ avem

\[
\label{l2star}
|\xi_i^{(m)} - \xi_i| \leq \varepsilon.
\]

Întrucîț $x_m=(\xi_i^{(m)})\in l^\infty$ reiese că $(x_m)$ este mărginit, i.e. există un număr real $k_m$ încît pentru orice $i$ avem $|\xi_i^{(m)}|\leq k_m$. Astfel aplicînd inegalitatea triunghiului obținem ($m>m_\varepsilon$)

\[
|\xi_i|=|\xi_i-\xi_i^{(m)}+\xi_i^{(m)}|\leq |\xi_i-\xi_i^{(m)}|+|\xi_i^{(m)}|\leq\varepsilon+k_m
\]

Inegalitatea de mai sus este adevărată pentru orice $i$ și din cauza că în membrul drept $i$ nu este prezent rezultă că șirul $(\xi_i)$ este mărginit, i.e. $x\in l^\infty$. De asemenea din \eqref{l2star} pentru $m>m_\varepsilon$ obținem

\[
d(x_m,x)=\sup_i |\xi_i^{(m)}-\xi_i|\leq \varepsilon.
\]

Rezulta că $x_m\to x$. Așadar $l^\infty$ este complet.

\end{proof}

\paragraph{Spațiul $c$} Spațiul $c$ este complet. 

\begin{proof}
Reamintim că $c$ este spațiul șirurilor convergente de numere reale sau complexe.
Metrica acestuia este cea indusă din $l^\infty$ deoarece $c$ este subspațiu al lui $l^\infty$.
Vom utiliza Teorema ??? , i.e. vom arăta că $c$ este un subspațiu închis în $l^\infty$.

Fie $x=(\xi_i)\in cl c$. Atunci există $x_m=(\xi_i^{(m)})\in c$ astfel încît $x_m\to x$ cînd $m\to\infty$. Atunci pentru orice $\varepsilon$ pozitiv, există un rang $n_{\varepsilon}$ astfel încît pentru toți termenii $x_m$ de rang mai mare decît $n_{\varepsilon}$ ($m>n_{\varepsilon}$) să avem:

\[
  d(x_m,x) = \sup_{i=1}^{\infty} |\xi_i^{(m)} - \eta_i| < \varepsilon
\]

și deoarece supremumul este luat de la elemente nenegative reiese că pentru orice $m>n_\varepsilon$ și $i=\overline{1,n}$ avem

\[
|\xi_i^{(m)} - \xi_i| < \varepsilon.
\]

În același timp $\xi_i^{(m)}$ sînt coordonatele elementelor din $c$, adică elemente de șiruri convergente. Deci este și Cauchy. Deci pentru orice $\varepsilon_1$ pozitiv, există un rang $n_{\varepsilon_1}$ astfel încît pentru toți termenii $\xi_i^{(m)}$ și $\xi_j^{(m)}$ de rang mai mare decît $n_{\varepsilon_1}$ ($i,j>n_{\varepsilon_1}$) să avem:

\[
|\xi_i^{(m)}-\xi_j^{(m)}|<\varepsilon_1
\]

Aplicăm inegalitatea triunghiului pentru orice $i,j>\varepsilon_1$:

\[
|\xi_i-\xi_j|<|\xi_i^{(m)}|+|\xi_i^{(m)}-\xi_j^{(m)}|+|\xi_j^{(m)}-\xi_j|<\varepsilon.
\]

Rezultă că $(\xi_i)$ este șir convergent, i.e. $x=(x_i)\in c$. Așadar $x\in cl c$.
\end{proof}

\paragraph{Spațiul $l^p$} Spatiul $l^p$ este complet, $1\leq p<\infty$.

\begin{proof}
Fie $(x_m)$ un șir Cauchy în $l^p$, unde $x_m=(\xi_1^{(m)},\xi_2^{(m)},...)$. Întrucîț metrica pe $l^p$ este definită prin

\[
  d(x,y) = (\Sigma_{i=1}^\infty |\xi_i - \eta_i|^p)^{\frac{1}{p}}
\]

unde $x=(\xi_i)$ și $y=(\eta_i)$.  și $(x_m)$ este Cauchy, pentru orice $\varepsilon$ pozitiv, există un rang $n_{\varepsilon}$ astfel încît pentru toți termenii $x_m$ și $x_p$ de rang mai mare decît $n_{\varepsilon}$ ($m,p>n_{\varepsilon}$) să avem:

\[
\label{l3}
d(x_m,x_k) = (\Sigma_{i=1}^\infty |\xi_i^{(m)} - \xi_i^{(k)}|^p)^{\frac{1}{p}} < \varepsilon.
\]

și deoarece este luat de la elemente nenegative reiese că pentru orice $m,k>n_\varepsilon$ și $i=\overline{1,n}$ avem

\[
\label{l4}
|\xi_i^{(m)} - \xi_i^{(k)}| < \varepsilon.
\]

Dacă fixăm $i$ din \ref{l4} reiese că $\xi_i^{(1)},\xi_i^{(2)},...$ este Cauchy. Și respectiv este convergent deoarece $\mathbb{R}$ și $\mathbb{C}$ sînt spații complete. Fie $\xi_i^{(m)}\to \xi_i$ cînd $m\to\infty$. Folosind aceste limite definim $x=(\xi_1, \xi_2,...)$. Vom arăta că $x\in l^p$ și că $x_m\to x$.

Din \ref{l3} (ridicînd la puterea $p$) avem pentru orice $m.k>n_\varepsilon$:

\[
\Sigma_{i=1}^r |\xi_i^{(m)} - \xi_i^{(k)}|^p < \varepsilon^p.
\]

Punînd $k\to\infty$ pentru $m>n_\varepsilon$ avem

\[
\Sigma_{i=1}^r |\xi_i^{(m)} - \xi_i|^p \leq \varepsilon^p.
\]

Acum punînd $r\to\infty$:

\[
\label{l5}
\Sigma_{i=1}^\infty |\xi_i^{(m)} - \xi_i|^p \leq \varepsilon^p.
\]

Ultima inegalitate arată că: $x_m-x=(\xi_i^{(m)}-\xi_i)\in l^p$. Întrucîț $x_m\in l^p$ din inegalitatea lui Minkowski reiese:

\[
x=x_m+(x-x_m)\in l^p.
\]

Mai mult, seria din \ref{l5} reprezintă $[d(x_m,x)]^p$ și deci \ref{l5} implică $x_m\to x$.

\end{proof}

\paragraph{Spațiul $C[a,b]$} Spațiul de funcții continue $C[a,b]$ este complet; $[a,b]$ interval închis în $\mathbb{R}$.

\begin{proof}
Fie $(x_m)$ un șir Cauchy în $C[a,b]$, unde $x_m=x_m(t)$, $t\in [a,b]$. Întrucîț metrica pe $C[a,b]$ este definită prin

\[
  d(x,y) = max_{t\in [a,b]} |x(t)-y(t)|
\]

unde $x=x(t)$ și $y=y(t)$.  și $(x_m)$ este Cauchy, pentru orice $\varepsilon$ pozitiv, există un rang $m_{\varepsilon}$ astfel încît pentru toți termenii $x_m$ și $x_p$ de rang mai mare decît $m_{\varepsilon}$ ($m,p>m_{\varepsilon}$) să avem:

\[
\label{l6}
d(x_m,x_p) = max_{t\in [a,b]} |x_m(t)-x_p(t)|<\varepsilon
\]

Pentru orice $t_0\in [a,b]$

\[
|x_m(t_0)-x_p(t_0)|<\varepsilon.
\]

Reiese că $x_1(t_0),x_2(t_0),...$ este un șit numeric Cauchy. Întrucît $\mathbb{R}$ este complet acest șir converge. Fie $x_m(t_0)\to x(t_0)$ cînd $m\to\infty$. În așa fel pentru orice $t\in[a,b]$ putem asocia un număr $x(t)$. Definește punctual o funcție $x$ pe $[a,b]$. Vom arata că $x\in C[a,b]$ și $x_m\to x$.

Din \ref{l6} cu $p\to\infty$

\[
max_{t\in [a,b]} |x_m(t)-x(t)|\leq \varepsilon
\]

Deci pentru orice $t$

\[
|x_m(t)-x(t)|\leq \varepsilon
\]

Ultima inegalitate arată că $(x_m(t))$ converge uniform către $x(t)$ pe $[a,b]$. Deoarece $x_m(t)$ sînt continue și convergența este uniformă rezultă că și $x(t)$ este continuă
\end{proof}


% Completarea spaţiilor metrice. Teorema Hausdorff.

%-------------------------------------------------------------------------------
\subsection{Completarea spațiilor metrice}
%-------------------------------------------------------------------------------

%-------------------------------------------------------------------------------
\subsection{Teorema Hausdorff}
%-------------------------------------------------------------------------------

% Teorema lui Cantor. Mulţimi rare. Teorema Baire.

%-------------------------------------------------------------------------------
\subsection{Teorema lui Cantor}
%-------------------------------------------------------------------------------

%-------------------------------------------------------------------------------
\subsection{Mulțimi rare}
%-------------------------------------------------------------------------------

%-------------------------------------------------------------------------------
\subsection{Teorema Baire}
%-------------------------------------------------------------------------------

% Aplicaţii de contracţie. Principiul punctului fix. Aplicații generalizate de  contracţie. Aplicaţii generalizate de  contracţie. Aplicaţii ale principiului  punctului fix.

%-------------------------------------------------------------------------------
\subsection{Contracții}
%-------------------------------------------------------------------------------

%-------------------------------------------------------------------------------
\subsection{Principiul punctului fix}
%-------------------------------------------------------------------------------

%-------------------------------------------------------------------------------
\subsection{Aplicaţii ale principiului  punctului fix}
%-------------------------------------------------------------------------------

%===============================================================================
\section{Săptămâna 9}
%===============================================================================

% Mulţimi compacte. Teorema Hausdorf  şi consecinţele ei.

%-------------------------------------------------------------------------------
\subsection{Mulţimi compacte}
%-------------------------------------------------------------------------------

%-------------------------------------------------------------------------------
\subsection{Teorema Hausdorf}
%-------------------------------------------------------------------------------

% Criterii de cmpacitate în C  a,b  Acoperiri. Teorema Borel. Funcţii continui    pe mulţimi compacte.

%-------------------------------------------------------------------------------
\subsection{Criterii de cmpacitate în $C[a,b]$}
%-------------------------------------------------------------------------------

%-------------------------------------------------------------------------------
\subsection{Teorema Borel}
%-------------------------------------------------------------------------------

%-------------------------------------------------------------------------------
\subsection{Funcții continui pe mulțimi compacte}
%-------------------------------------------------------------------------------

% Spaţii liniare normate. Subspaţii. Sume directe şi subspaţii.

%-------------------------------------------------------------------------------
\subsection{Norma}
%-------------------------------------------------------------------------------

% Spaţii Banah. Spaţii cât.

%-------------------------------------------------------------------------------
\subsection{Spații Banach}
%-------------------------------------------------------------------------------

%-------------------------------------------------------------------------------
\subsection{Spații cît}
%-------------------------------------------------------------------------------

%===============================================================================
\section{Săptămâna 10}
%===============================================================================

% Izomorfizmul spaţiilor normate finit dimensionale. Spaţiile L    şi L

%-------------------------------------------------------------------------------
\subsection{Izomorfizmul spaţiilor normate finit dimensionale}
%-------------------------------------------------------------------------------

%-------------------------------------------------------------------------------
\subsection{Spaţiile $L$ și $L$}
%-------------------------------------------------------------------------------

% Spaţii  Hilbert. Spaţii prehilbertiene.

%-------------------------------------------------------------------------------
\subsection{Spaţii  Hilbert}
%-------------------------------------------------------------------------------

%-------------------------------------------------------------------------------
\subsection{Spaţii prehilbertiene}
%-------------------------------------------------------------------------------

% Operatori liniari şi mărginiţi. Exemple.

%-------------------------------------------------------------------------------
\subsection{Operatori liniari}
%-------------------------------------------------------------------------------

%-------------------------------------------------------------------------------
\subsection{Operatori mărginiţi}
%-------------------------------------------------------------------------------

% Norma operatorului  liniar. Exemple.

%-------------------------------------------------------------------------------
\subsection{Norma operatorului liniar}
%-------------------------------------------------------------------------------

%===============================================================================
\section{Săptămâna 11}
%===============================================================================

% Spaţiul operatoriloi liniari. Principiul mărginirii uniforme.

%-------------------------------------------------------------------------------
\subsection{Spațiul operatorilor liniari}
%-------------------------------------------------------------------------------

%-------------------------------------------------------------------------------
\subsection{Principiul mărginirii uniforme}
%-------------------------------------------------------------------------------

% Convergenţa tare a şirului de operatori. Prelungirea operatorilor liniari.

%-------------------------------------------------------------------------------
\subsection{Convergenţa tare a şirului de operatori}
%-------------------------------------------------------------------------------

%-------------------------------------------------------------------------------
\subsection{Prelungirea operatorilor liniari}
%-------------------------------------------------------------------------------

% Funcţionale liniare. Exemple.

%-------------------------------------------------------------------------------
\subsection{Funcționale liniare}
%-------------------------------------------------------------------------------

% Prelungirea funcţionalelor liniare. Teorema Hahn-Banach.

%-------------------------------------------------------------------------------
\subsection{Prelungirea funcționalelor liniare}
%-------------------------------------------------------------------------------

%-------------------------------------------------------------------------------
\subsection{Teorema Hahn-Banach}
%-------------------------------------------------------------------------------

%===============================================================================
\section{Săptămâna 12}
%===============================================================================

% Forma generală a funcţionalelor liniare şi mărginite în unele spaţii finitdimensionale.

%-------------------------------------------------------------------------------
\subsection{Forma generală a funcţionalelor liniare}
%-------------------------------------------------------------------------------

% Forma generală a funcţionalelor în spaţiile Ca,b şi  Lp  a,b

%-------------------------------------------------------------------------------
\subsection{Forma generală a funcţionalelor în spaţiile }
%-------------------------------------------------------------------------------

% Convergenţa slabă. Teorema Polya-Steclov.

%-------------------------------------------------------------------------------
\subsection{Convergenţa slabă}
%-------------------------------------------------------------------------------

%-------------------------------------------------------------------------------
\subsection{Teorema Polya-Steclov}
%-------------------------------------------------------------------------------

% Măsura mărimilor elementare. Măsura Lebesgue a mulţimilor plane.

%-------------------------------------------------------------------------------
\subsection{Măsura mărimilor elementare}
%-------------------------------------------------------------------------------

%-------------------------------------------------------------------------------
\subsection{Măsura Lebesgue a mulțimilor plane}
%-------------------------------------------------------------------------------

%===============================================================================
\section{Săptămâna 13}
%===============================================================================

% Noţiuni generale despre măsură. Prelungirea măsurii de pe un inel. - aditivitatea măsurii. Prelungirea Lebesque  a măsurii.

%-------------------------------------------------------------------------------
\subsection{Măsura}
%-------------------------------------------------------------------------------

%-------------------------------------------------------------------------------
\subsection{Prelungirea Lebesque a măsurii}
%-------------------------------------------------------------------------------

% Funcţiile măsurabile. Teorema Egorov şi teorema Luzin. 

%-------------------------------------------------------------------------------
\subsection{Funcțiile măsurabile}
%-------------------------------------------------------------------------------

%-------------------------------------------------------------------------------
\subsection{Teorema Egorov şi teorema Luzin}
%-------------------------------------------------------------------------------

% Funcţii simple . Integrala Lebesque

%-------------------------------------------------------------------------------
\subsection{Funcții simple}
%-------------------------------------------------------------------------------

%-------------------------------------------------------------------------------
\subsection{Integrala Lebesgue}
%-------------------------------------------------------------------------------

% -aditivitate şi trecerea la limită sub semnul integralei Lebesque.

%-------------------------------------------------------------------------------
\subsection{Trecerea la limită sub semnul integralei Lebesgue}
%-------------------------------------------------------------------------------

%===============================================================================
\section{Săptămâna 14}
%===============================================================================

% Legătura între integrala Lebesque şi integrala Riemann. Funcţii cu variaţie mărginită.

%-------------------------------------------------------------------------------
\subsection{Legătura între integrala Lebesgue şi integrala Riemann}
%-------------------------------------------------------------------------------

% Produsul sistemului de mulţimi, produsul măsurilor.

%-------------------------------------------------------------------------------
\subsection{Produsul măsurilor}
%-------------------------------------------------------------------------------

% Măsura Stiltes. Integrala Lebesque-Stiltes. Integrala Riemann-Stiltes.

%-------------------------------------------------------------------------------
\subsection{Măsura Stiltjes}
%-------------------------------------------------------------------------------

% Trecerea la limită sub semnul integralei Stiltes.

%-------------------------------------------------------------------------------
\subsection{Trecerea la limită sub semnul integralei Stiltjes}
%-------------------------------------------------------------------------------

%===============================================================================
\section{Săptămâna 15}
%===============================================================================

\end{document}